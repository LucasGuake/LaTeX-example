% 文档类:article(短文),字体大小12pt,A4纸
\documentclass[12pt,a4paper]{article}

% 导入必要宏包(解决中文、图表、数学公式等问题)
\usepackage{ctex}         % 支持中文
\usepackage{amsmath}      % 高级数学公式
\usepackage{graphicx}     % 插入图片
\usepackage{enumerate}    % 自定义列表
\usepackage{hyperref}     % 超链接(目录、参考文献)

% 文档标题、作者、日期(\maketitle会自动生成标题区)
\title{LaTeX 基础使用示例}
\author{王五 \quad 张三 \\ 某大学某学院}  % \quad 用于作者间空格
\date{\today}  % \today 自动填充当前日期,也可手动写如2025年10月16日

% 开始文档内容
\begin{document}

% 生成标题区(必须在\begin{document}之后)
\maketitle

% 摘要环境(学术文档常用)
\begin{abstract}
本文以示例形式介绍LaTeX的基础语法,包括文档结构、数学公式、列表、图片插入和参考文献管理。通过该示例,读者可快速掌握LaTeX的核心使用方法,满足学术论文或技术文档的编写需求。
\end{abstract}

% 生成目录(自动抓取章节标题)
\tableofcontents
\clearpage  % 强制目录页单独一页,内容从新页开始

% 1级章节(section)
\section{引言}
LaTeX是一种基于TeX的排版系统,特别适合包含大量数学公式、图表的学术文档。与Word等可视化排版工具不同,LaTeX通过“代码”控制格式,能实现更统一、专业的排版效果,广泛应用于数学、物理、计算机等领域。

% 2级章节(subsection)
\subsection{LaTeX的优势}
LaTeX的核心优势体现在以下3点:
\begin{enumerate}[1.]  % 自定义列表编号为“1.”“2.”
    \item 自动处理格式:无需手动调整段落、标题样式,专注内容创作。
    \item 数学公式支持:通过amsmath等宏包,轻松编写复杂公式。
    \item 跨平台兼容:生成的PDF文档在任何设备上格式一致,无错乱风险。
\end{enumerate}

% 1级章节:数学公式示例
\section{数学公式}
LaTeX支持“行内公式”和“独立公式”两种形式,以下为常见示例。

\subsection{行内公式}
行内公式嵌入正文,用\$符号包裹,例如:线性方程\( y = ax + b \),其中\( a \)为斜率,\( b \)为截距。

\subsection{独立公式}
独立公式单独成行,用equation环境包裹(自动编号):
\begin{equation}  % 公式自动编号为(1)
    f(x) = \int_{0}^{1} \frac{\sin(t)}{t + x} dt
\end{equation}
公式(1)为一个定积分表达式,\int表示积分符号,_{0}和^{1}分别指定积分上下限。

% 1级章节:图片插入示例
\section{图片插入}
插入图片需先准备图片文件(如example.png),并放在与.tex文件同一目录下。以下为单图插入示例:

\begin{figure}[h]  % [h]表示图片优先放在当前位置(here)
    \centering  % 图片居中
    \includegraphics[width=0.6\textwidth]{example.png}  % 宽度设为页面60%
    \caption{示例图片}  % 图片标题(自动编号)
    \label{fig:example}  % 图片标签,用于后文引用
\end{figure}

如上图\ref{fig:example}所示,通过graphicx宏包的\includegraphics命令可灵活调整图片大小和位置。

% 1级章节:参考文献示例
\section{参考文献}
参考文献用thebibliography环境管理,通过\cite{key}引用。

\begin{thebibliography}{9}  % {9}表示参考文献最多9篇(可调整为99)
    \bibitem{latex-guide}  % 引用标识(如\cite{latex-guide})
    刘海洋. LaTeX入门[M]. 北京:电子工业出版社,2013.
    
    \bibitem{amsmath-doc}
    Leslie Lamport. A Document Preparation System[M]. Reading: Addison-Wesley, 1994.
\end{thebibliography}

% 结束文档
\end{document}